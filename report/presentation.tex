% !TEX TS-program = pdflatex
% !TEX encoding = UTF-8 Unicode

% This file is a template using the "beamer" package to create slides for a talk or presentation
% - Talk at a conference/colloquium.
% - Talk length is about 20min.
% - Style is ornate.

\documentclass{beamer}

\mode<presentation>
{
  \usetheme{Marburg}
  \setbeamercovered{transparent}
  % or whatever (possibly just delete it)
}


\usepackage[english]{babel}
\usepackage[utf8]{inputenc}

%\usepackage{times}
%\usepackage[T1]{fontenc}
% Or whatever. Note that the encoding and the font should match. If T1
% does not look nice, try deleting the line with the fontenc.

%NB: gebruik \dops{} of \dops~ om netjes een whitespace erachter te krijgen in lopende tekst
\newcommand{\dops}[0]{DOP$ ^*$}
\newcommand{\ddop}[0]{Double-DOP}


\title{Doubling \dops}
\subtitle{ A comparison of \ddop{} and \dops{}}
\author[Kruit, Veldhoen]{Benno Kruit\and Sara Veldhoen{\\\small Supervised by: \\Andras van Cranenburg \and Khalil Sima'an}}

\institute{University of Amsterdam (UvA)}
\date{Project AI, January 2014}



% Delete this, if you do not want the table of contents to pop up at
% the beginning of each subsection:
\AtBeginSubsection[]
{
  \begin{frame}<beamer>{Outline}
    \tableofcontents[currentsection,currentsubsection]
  \end{frame}
}


% If you wish to uncover everything in a step-wise fashion, uncomment the following command: 
%\beamerdefaultoverlayspecification{<+->}


\begin{document}

\begin{frame}
  \titlepage
\end{frame}

\begin{frame}{Outline}
  \tableofcontents   %[pausesections]
\end{frame}

\section{Data Oriented Parsing}

\subsection{Introduction to DOP}

\begin{frame}{Parsing}%{Subtitle}
  % - A title should summarize the slide in an understandable fashion
  %   for anyone how does not follow everything on the slide itself.

  \begin{itemize}
  \item input: sentence \pause
  \item output: constituent tree
  \end{itemize}
\end{frame}

\begin{frame}{Grammar}
A grammar describes:
\begin{itemize}
\item how trees can be built
\begin{itemize} 
\item CFG's - elementary rules
\item TSG's  - larger units: \emph{fragments}
\end{itemize}
\item how likely constructions are: \emph{probabilistic} grammars
\begin{itemize} 
\item PCFG's - independence 
\item TSG's  - derivations
\end{itemize}

\end{itemize}
\end{frame}


\subsection{Bias and Consistency}

\begin{frame}{Title}
\end{frame}

\section{\ddop{} and \dops{}: a comparison}

\subsection{Introduction to \ddop{} and \dops{}}
\begin{frame}{\ddop{}}
\begin{itemize}
\item Extraction: Maximal Overlap
\item Estimation: relative frequency 
\item Coverage: PCFG rules
\end{itemize}
\end{frame}

\begin{frame}{\dops{}}
\begin{itemize}
\item Held-out estimation - $HC$ and $EC$
\item Extraction: Shortest derivations
\item Estimation: relative frequency \emph{in shortest derivations}
\item Coverage: smoothing PCFG rules with probability $p_{unkn}$
\end{itemize}
\end{frame}

\subsection{Comparison}
\begin{frame}{Comparison}
\begin{itemize}
\item Shortest derivations or Maximal overlap
\item Held-out estimation or one vs. the rest
\end{itemize}
\end{frame}

\subsection{Experiments}


\section{Results}

\subsection{Analyzing grammars}

\begin{frame}{Maximal overlap $\leftrightarrow$ shortest derivation}
\end{frame}

\begin{frame}{Split $\leftrightarrow$ one vs. the rest}
\end{frame}

\subsection{Parsing Performance}
\begin{frame}{F1 scores}
\end{frame}


\section*{Summary}

\begin{frame}{Summary}

  % Keep the summary *very short*.
  \begin{itemize}
  \item
    The \alert{first main message} of your talk in one or two lines.
  \item
    The \alert{second main message} of your talk in one or two lines.
  \item
    Perhaps a \alert{third message}, but not more than that.
  \end{itemize}
  
  % The following outlook is optional.
  \vskip0pt plus.5fill
  \begin{itemize}
  \item
    Outlook
    \begin{itemize}
    \item
      Something you haven't solved.
    \item
      Something else you haven't solved.
    \end{itemize}
  \end{itemize}
\end{frame}

\bibliographystyle{acl2014}
\bibliography{bibliography}


%
%
%% All of the following is optional and typically not needed. 
%\appendix
%\section<presentation>*{\appendixname}
%\subsection<presentation>*{For Further Reading}
%
%\begin{frame}[allowframebreaks]
%  \frametitle<presentation>{For Further Reading}
%    
%  \begin{thebibliography}{10}
%    
%  \beamertemplatebookbibitems
%  % Start with overview books.

  % \bibitem{Author1990}
  %   A.~Author.
  %   \newblock {\em Handbook of Everything}.
  %   \newblock Some Press, 1990.
 
    
  % \beamertemplatearticlebibitems
  % % Followed by interesting articles. Keep the list short. 

  % \bibitem{Someone2000}
  %   S.~Someone.
  %   \newblock On this and that.
  %   \newblock {\em Journal of This and That}, 2(1):50--100,
  %   2000.
  % \end{thebibliography}
% \end{frame}

\end{document}


