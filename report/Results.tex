\section{Results and analysis}\label{sec:Results}

There are three grammars we analyze: maximal overlap extraction with both split and full estimation, and shortest derivation with split estimation.The split estimation was done by interpolating the estimates produced by ten random splits.  All grammars have been smoothed with PCFG rules to maximize coverage. In the maximal overlap approach, this is done internally as defined in \ddop{}. In the shortest derivation apporach, we redistribute a proportion $p_{unkn}$ of the probability mass over the PCFG grammar. $p_{unkn}$ was found to be $1.41\times 10^{-3}$ for our dataset.%0.00140590480016

%SIZE OF THE GRAMMARs

\subsection{Parsing performance}

\begin{table*}
\center

\begin{tabular}{l | ccc}
&Maximal~Overlap&Maximal~Overlap&Shortest~Derivation\\
& 1~vs.~Rest& Split&Split\\\hline
labeled recall&90,97&90.14&84.90\\
labeled precision&90.25&90.10&84.39\\
labeled f-measure&90.61&90.12&84.64\\
exact match&53.27&49.53&39.25\\
\end{tabular}

\caption{Results for 321 sentences of length$<16$}

\label{t:performance}
\end{table*}

Table \ref{t:performance} shows the parsing performance for the three grammars we constructed. Note that the POS-tags were passed to the parser in all cases, so tagging accuracy was 100\% and is omitted from this table. %Leaf-ancestor?

Both Maximal Overlap grammars perform much better than the Shortest Derivation one, in spite of the latter being consistent. 
One possible explanation, is the smoothing we conducted. Recall that the coverage of the Maximal Overlap approach is catered in a rather natural way, by extending the symbolic grammar with all PCFG rules from the treebank and treating them like the other fragments in the estimation. In the case of Shortest Derivation however, the coverage was a bit more artificial. $p_{unkn}$ was computed over all folds and used to redistribute weights over a classical PCFG constructed from the entire treebank. 

%Shortest derivation: much smaller grammar

\subsection{Pairwise comparison of the grammars}
In each plot, two grammars are compared to each other. The fragments are presented in a scatter plot, with the weights assigned by the two grammars along the axes. The weights are best visualized on a logarithmic scale. However, it is also informative to see those fragments with value zero. Therefore, the first interval ($[0,10^{-6}]$) is linear, while the rest of the plot is logarithmic. 
The difference between grammars is represented by the distance of the points to the \emph{identity line} $x=y$.
The color corresponds to some feature, e.g. the depth of the fragment. The color mapping is also logarithmic. 

% 1 figuur:
% SDS MOS depth (both split)
% SDS MOF depth (original DDOP en DOP*)
% MOS MOF depth (split vs full)
\begin{figure*}
\center
\begin{subfigure}{0.32\textwidth}
\includegraphics[width=\linewidth,trim=0.5cm 0cm 2.5cm 0.5cm, clip=true]{../data/plots/0.png}
\caption{}
\label{f:SDS-MOS}
\end{subfigure}
\begin{subfigure}{0.32\textwidth}
\includegraphics[width=\linewidth,trim=0.5cm 0cm 2.5cm 0.5cm, clip=true]{../data/plots/1.png}
\caption{}
\label{f:SDS-MOF}
\end{subfigure}
\begin{subfigure}{0.32\textwidth}
\includegraphics[width=\linewidth,trim=0.5cm 0cm 2.5cm 0.5cm, clip=true]{../data/plots/2.png}
\caption{}
\label{f:MOS-MOF}
\end{subfigure}

\caption{Comparing three grammars by depth of the fragments}
\label{f:depth3}
\end{figure*}

\paragraph{Depth of the fragments}
In figure \ref{f:depth3}, the color of the points in the scatter plot refer to the depth of the fragments. Depth is a common measure of fragment size. Johnson shows in \shortcite{johnson2002} that the original DOP1 had a bias towards larger fragments. 

Plot \ref{f:MOS-MOF} illustrates the effect of the split estimation as compared to the one vs. rest. We see a remarkable separation of fragments with larger depth (lighter color) below the identity line, and fragments with smaller depth above it. This indicates that the split estimation tends to reduce the bias towards larger fragments.

%Diepe fragmenten zijn zwaarder in :
%	SD dan in MOS
%	SD dan MO
%	MO dan MOS (scherp)

%Brede fragmenten zijn zwaarder in:
% 	SD dan MOS (punt: CFG rules)
%	MO dan MOS (scherp)









%SDS MOS width


















